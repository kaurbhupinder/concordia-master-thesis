\begin{abstract}
Software containers greatly facilitate the deployment and reproducibility
of scientific data analyses on high-performance computing clusters
(HPC). However, container images often contain outdated or unnecessary
software packages, which increases the number of security vulnerabilities
in the images and widens the attack surface of the infrastructure. This
thesis presents a vulnerability analysis of container images for scientific
data analysis. We compare results obtained with four vulnerability
scanners, focusing on the use case of neuroscience data analysis, and
quantifying the effect of image update and minification on the number of
vulnerabilities. We find that container images used for neuroscience data analysis
contain hundreds of vulnerabilities, that software updates remove about two
thirds of these vulnerabilities, and that removing unused packages is also
effective. We conclude with recommendations on how to build container
images with a reduced amount of vulnerabilities.
 
\end{abstract}
