\chapter{Introduction}
Containers are very lightweight, flexible, and more resource-efficient than
virtual machines. They provide an operating system (OS) level virtualization,
which removes the overhead of having an extra OS layer. However, containers
tightly integrate with the host due to the sharing of the kernel, which raises
security concerns. In this chapter, we provide a brief introduction of containers,
mainly discussing two popular containerization techniques Docker and Singularity.

\section{Containers}
OS-level virtualization initially started in 1979 when Chroot was introduced.
Chroot is a Unix system call, which changes the root directory
of a process and its children processes. A chrooted process cannot access files
outside the designated directory.
As the time passed, this virtualization got evolved and took different forms of
FreeBSD jails, Linux VServer project, and Solaris containers. However, in 2008,
with the introduction of Linux containers (LXC) OS-level virtualization reached
to an another level of popularity. LXC gained advantage over
other Linux-based projects due to the integration of virtualization features into
the upstream Linux kernel, which removed the need of applying patches to the
Linux kernel and recompilation~\cite{reshetova2014security}. 
LXC relies on Linux kernel’s control groups (cgroups)
and namespaces. Cgroups control the usage of system resources (memory, CPU,
network, etc) among a group of processes. Namespaces define what a process
can see by providing an isolated view of the operating system. For example,
PID namespace isolate processes running in a container~\cite{bernstein2014containers}.

\subsection{Docker}
Docker was introduced in 2013 and has gained popularity since then. Docker
is built on top of LXC. It automates application deployment, in addition to
providing isolated environment. Docker allows software engineers to develop an
application only once, on one system, and then use it on other systems without
much effort. With the help of Docker, developers pack the application with
all its dependencies, libraries, and tools, which can be run on any Linux server
later. Docker containers share boot file-system and kernel with the host, which
raises security concerns. Another issue is that Docker is not supported by
traditional high-performance computing (HPC) resources. This is mainly due to the
default configuration of Docker, which runs containers as root. Consequently, it
becomes hard for system administrators to keep a record of which user is doing
what~\cite{Containers2017, anderson2015docker}.

\subsubsection{Docker Daemon}
Docker engine is known as Docker daemon. It is a service that runs on the
host operating system and is responsible for managing all Docker containers.
It also allocates the host’s resources to Docker containers. Like LXC, Docker
daemon also uses \textit{cgroups} and \textit{namespaces} of Linux, and hence only support
Linux operating system. Docker daemon requires root access to operate, and
every container run by Docker is spawned as a child of root-owned Docker 
daemon. Therefore, users can gain escalated privileges by coercing Docker daemon,
which creates serious security risks in shared environments. As a consequence,
Docker is not supported in multi-user environments.

\subsubsection{Docker Images}
A Docker image is a package which contains an application along with all 
required dependencies and libraries. All the steps of creating a particular image
can be saved in a Docker file. Using Dockerfile, there are two ways to create
a Docker image. The first is to use an existing Docker image as a base image
and then build a new image from that one, by making changes in the filesystem,
which is saved as a new layer of the Docker image. The second way is to build
a new Docker image from scratch. Docker images are organized as a series of
\textit{layers} which are stacked on top of each other. Each layer consists of a filesystem
diff that is introduced due to the changes made on the layer below it. All layers
of the Docker image are identified by unique \textit{layerIDs}. All these layers stacked
together gives a unified and complete view of a Docker image. These layers are
compressed into a single image and can be pushed to the Docker Hub.

\subsubsection{Docker Hub}
Docker Hub is a repository used by Docker to keep all Docker
images in one central place. Docker images on Docker Hub can be
private or public. Docker Hub also proposes automated builds and
webhooks. Docker webhook is an HTTP call-back triggered by a
\textit{push} on external code repository. Automated builds,
also known as autobuilds, automatically build Docker images
from the source code located on external code repository (e.g. GitHub,
BitBucket, etc.) and then host them automatically on Docker Hub.
During the setup of Docker automated builds, a list of branches
and tags, that the user wants to build into Docker images is created.
When source code is pushed to a code repository (like GitHub) for
these listed image tags, \textit{push} uses Docker webhook to build a new
Docker image~\cite{dockerdoc_2019}. The created image is then
automatically pushed to Docker Hub. After the automated build,
Docker sends HTTP request to a reachable Docker host on the
Internet to notify it about the availability of new image,
which pulls the built image and restarts container on the
new image through Docker webhooks~\cite{martin2018docker,dockerdoc2__2019}.

\subsection{Singularity}

Singularity was introduced in 2016. It offers mobility of compute by
facilitating portable environments through a single image file, which
makes Singularity containers different from Docker's layered containers.
Once Singularity image is created, it is hashed by using SHA256 hashing.
Hence, it cannot be changed once it is created, consequently, it can be
used for reproducing the results of scientific experiments. Singularity
was introduced with the main goal of providing support for multi-user
environment \cite{Singularity}, where Docker failed to provide such
support. Main features provided by Singularity are mobility of compute
and reproducibility.

Mobility of compute is defined as creating and maintaining a workflow
on a local machine that can be easily ported on other hosts, Linux
distributions, and/or cloud service providers. To achieve this,
Singularity packs everything that is required for the application
to run in a distributable image. Then that image can be copied,
archived, and shared. Moreover, Singularity containers are easily
portable across different versions of the C library and different
kernel implementations \cite{kurtzer2017singularity}.

Same features of Singularity, which are used for mobility of compute,
facilitate reproducibility as well. Hash of the image is also stored
along with the image \cite{kurtzer2017singularity}.


\subsubsection{Singularity Images}

Singularity takes snapshot, locks, and archives the developed application,
which is known as Singularity image \cite{kurtzer2016singularity}.
Due to image hashing, Singularity does not contain image layers, unlike
Docker images. While publishing experimental results, authors can also
publish Singularity image along with its hash, which allow other
researchers to verify results.


\subsubsection{Singularity Hub}

Singularity Hub is a central repository provided by Singularity to keep
Singularity images and their hash. It is designed to facilitate
reproducible research and publications. Singularity Hub can be easily
cited in the publications and hence other researchers can easily
replicate conducted experiments. Singularity Hub connects with
GitHub for automatic builds of Singularity images from the source
code present at GitHub.

\section{Thesis Outline}
This thesis consists of four main parts divided into chapters.
Chapter~\ref{chapter2} provides literature survey on the security of containers. 
In accordance with that, it provides a list of security requirements that 
every OS-level virtualization technique should fulfill, and hence based on 
these requirements it investigates security of containers. This chapter 
also points out strong features and security weaknesses of containers.
Chapter~\ref{chapter3} presents various experiments done to analyze security 
of Docker and Singularity images used in two containerization frameworks 
used in neuroscience. Additionally, it 
provides results of these experiments.
Finally, Chapter~\ref{chapter4} concludes this thesis and highlights the future work.
